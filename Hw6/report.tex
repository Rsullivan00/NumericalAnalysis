\documentclass{article}

\usepackage{fancyhdr}
\usepackage{extramarks}
\usepackage{amsmath}
\usepackage{amsthm}
\usepackage{amsfonts}
\usepackage{tikz}
\usepackage[plain]{algorithm}
\usepackage{algpseudocode}
\usepackage{listings}
\usepackage{enumerate}
\lstset{breaklines=true}

\usetikzlibrary{automata,positioning}

%
% Basic Document Settings
%

\topmargin=-0.45in
\evensidemargin=0in
\oddsidemargin=0in
\textwidth=6.5in
\textheight=9.0in
\headsep=0.25in

\linespread{1.1}

\pagestyle{fancy}
\lhead{\hmwkAuthorName}
\chead{\hmwkClass\ (\hmwkClassInstructor): \hmwkTitle}
\rhead{\firstxmark}
\lfoot{\lastxmark}
\cfoot{\thepage}

\renewcommand\headrulewidth{0.4pt}
\renewcommand\footrulewidth{0.4pt}

\setlength\parindent{0pt}

%
% Create Problem Sections
%

\newcommand{\enterProblemHeader}[1]{
    \nobreak\extramarks{}{Problem \arabic{#1} continued on next page\ldots}\nobreak{}
    \nobreak\extramarks{Problem \arabic{#1} (continued)}{Problem \arabic{#1} continued on next page\ldots}\nobreak{}
}

\newcommand{\exitProblemHeader}[1]{
    \nobreak\extramarks{Problem \arabic{#1} (continued)}{Problem \arabic{#1} continued on next page\ldots}\nobreak{}
    \stepcounter{#1}
    \nobreak\extramarks{Problem \arabic{#1}}{}\nobreak{}
}

\setcounter{secnumdepth}{0}
\newcounter{partCounter}
\newcounter{homeworkProblemCounter}
\setcounter{homeworkProblemCounter}{1}
\nobreak\extramarks{Problem \arabic{homeworkProblemCounter}}{}\nobreak{}

%
% Homework Problem Environment
%
% This environment takes an optional argument. When given, it will adjust the
% problem counter. This is useful for when the problems given for your
% assignment aren't sequential. See the last 3 problems of this template for an
% example.
%
\newenvironment{homeworkProblem}[1][-1]{
    \ifnum#1>0
        \setcounter{homeworkProblemCounter}{#1}
    \fi
    \section{Problem \arabic{homeworkProblemCounter}}
    \setcounter{partCounter}{1}
    \enterProblemHeader{homeworkProblemCounter}
}{
    \exitProblemHeader{homeworkProblemCounter}
}

%
% Homework Details
%   - Title
%   - Due date
%   - Class
%   - Section/Time
%   - Instructor
%   - Author
%

\newcommand{\hmwkTitle}{Homework\ \#6}
\newcommand{\hmwkDueDate}{February 27, 2015}
\newcommand{\hmwkClass}{Numerical Analysis}
\newcommand{\hmwkClassTime}{MWF 2:15}
\newcommand{\hmwkClassInstructor}{Professor Mohler}
\newcommand{\hmwkAuthorName}{Rick Sullivan}

%
% Title Page
%

\title{
    \vspace{2in}
    \textmd{\textbf{\hmwkClass:\ \hmwkTitle}}\\
    \normalsize\vspace{0.1in}\small{Due\ on\ \hmwkDueDate}\\
    \vspace{0.1in}\large{\textit{\hmwkClassInstructor\ \hmwkClassTime}}
    \vspace{3in}
}

\author{\textbf{\hmwkAuthorName}}
\date{}

\renewcommand{\part}[1]{\textbf{\large Part \Alph{partCounter}}\stepcounter{partCounter}\\}

%
% Various Helper Commands
%

% Useful for algorithms
\newcommand{\alg}[1]{\textsc{\bfseries \footnotesize #1}}

% For derivatives
\newcommand{\deriv}[1]{\frac{\mathrm{d}}{\mathrm{d}x} (#1)}

% For partial derivatives
\newcommand{\pderiv}[2]{\frac{\partial}{\partial #1} (#2)}

% Integral dx
\newcommand{\dx}{\mathrm{d}x}

% Alias for the Solution section header
\newcommand{\solution}{\textbf{\large Solution}}

% Probability commands: Expectation, Variance, Covariance, Bias
\newcommand{\E}{\mathrm{E}}
\newcommand{\Var}{\mathrm{Var}}
\newcommand{\Cov}{\mathrm{Cov}}
\newcommand{\Bias}{\mathrm{Bias}}

\begin{document}

\maketitle

\pagebreak

\begin{homeworkProblem}
Find the first three iterations obtained by the Power method applied to the matrix
\[
    \begin{bmatrix}
        1&1&1\\
        1&1&0\\
        1&0&1
    \end{bmatrix}
\]
Use \(x^{(0)} = (-1,0,1)^t\).
\\

\textbf{Solution}
\\

\[
    \begin{split}
        \vec{x} &= A\vec{x}\\
        &= 
        \begin{bmatrix}
            1&1&1\\
            1&1&0\\
            1&0&1
        \end{bmatrix}
        \begin{bmatrix}
            x_1\\
            x_2\\
            x_3
        \end{bmatrix}
        \\
        &=
        \begin{bmatrix}
            x_1 + x_2 + x_3\\
            x_1 + x_2\\
            x_1 + x_3
        \end{bmatrix}
        \\
        &= 
        \begin{bmatrix}
            0\\
            -1\\
            0
        \end{bmatrix}
        \\
        \vec{x^{(1)}} &= \vec{x}/||\vec{x}||_\infty\\
        &= 
        \begin{bmatrix}
            0\\
            -1\\
            0
        \end{bmatrix}
        \\[4ex]
        \vec{x} &= 
        \begin{bmatrix}
            -1\\
            -1\\
            0
        \end{bmatrix}
        \\
        \vec{x^{(2)}} &= \vec{x}/||\vec{x}||_\infty\\
        &= 
        \begin{bmatrix}
            -1\\
            -1\\
            0
        \end{bmatrix}
        \\[4ex]
        \vec{x} &= 
        \begin{bmatrix}
            -1 -1 + 0\\
            -1 -1\\
            -1 + 0 
        \end{bmatrix}
        =\begin{bmatrix}
            -2\\
            -2\\
            -1 
        \end{bmatrix}
        \\
        \vec{x^{(3)}} &= \vec{x}/||\vec{x}||_\infty\\
        &= 
        \begin{bmatrix}
            -2\\
            -2\\
            -1 
        \end{bmatrix}/2
        \\
        &= 
        \begin{bmatrix}
            -1\\
            -1\\
            -1/2 
        \end{bmatrix}
    \end{split}
\]
\end{homeworkProblem}
\begin{homeworkProblem}
    Determine a singular value decomposition for the matrix 
 \[
    \begin{bmatrix}
        0&1&1\\
        0&1&0\\
        1&1&0\\
        0&1&0\\
        1&0&1\\
    \end{bmatrix}
\]
\\

\textbf{Solution}
\\

Using the python script
\lstinputlisting[language=python]{svd.py}
we get the following output:
\lstinputlisting{svd_out.txt}

To create a \(5\times1\), \(1\times1\), and a \(1\times3\) matrix whose product is approximately A, we can just take elements in the manner we discussed in class from our solution provided by the script. This will give us the matrices

\[
    \begin{bmatrix}
        -0.547722558\\
        -0.365148372\\
        -0.547722558\\
        -0.365148372\\
        -0.365148372
    \end{bmatrix}
    ,
    \begin{bmatrix}
        2.23505798
    \end{bmatrix}
    ,
    \begin{bmatrix}
        -0.408248290 & -0.816496581 & -0.408248290\\
    \end{bmatrix}
\]

The product of these matrices is the matrix
\[
    \begin{bmatrix}
        0.5 & 1 & 0.5\\
        0.33333 & 0.66667 & 0.33333\\
        0.5 & 1 & 0.5\\
        0.33333 & 0.66667 & 0.33333\\
        0.33333 & 0.66667 & 0.33333\\
    \end{bmatrix}
\]

\end{homeworkProblem}

%\pagebreak
%\section{Appendix}

\end{document}
