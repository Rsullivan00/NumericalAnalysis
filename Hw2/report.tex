\documentclass{article}

\usepackage{fancyhdr}
\usepackage{extramarks}
\usepackage{amsmath}
\usepackage{amsthm}
\usepackage{amsfonts}
\usepackage{tikz}
\usepackage[plain]{algorithm}
\usepackage{algpseudocode}
\usepackage{listings}
\usepackage{enumerate}
\lstset{breaklines=true}

\usetikzlibrary{automata,positioning}

%
% Basic Document Settings
%

\topmargin=-0.45in
\evensidemargin=0in
\oddsidemargin=0in
\textwidth=6.5in
\textheight=9.0in
\headsep=0.25in

\linespread{1.1}

\pagestyle{fancy}
\lhead{\hmwkAuthorName}
\chead{\hmwkClass\ (\hmwkClassInstructor): \hmwkTitle}
\rhead{\firstxmark}
\lfoot{\lastxmark}
\cfoot{\thepage}

\renewcommand\headrulewidth{0.4pt}
\renewcommand\footrulewidth{0.4pt}

\setlength\parindent{0pt}

%
% Create Problem Sections
%

\newcommand{\enterProblemHeader}[1]{
    \nobreak\extramarks{}{Problem \arabic{#1} continued on next page\ldots}\nobreak{}
    \nobreak\extramarks{Problem \arabic{#1} (continued)}{Problem \arabic{#1} continued on next page\ldots}\nobreak{}
}

\newcommand{\exitProblemHeader}[1]{
    \nobreak\extramarks{Problem \arabic{#1} (continued)}{Problem \arabic{#1} continued on next page\ldots}\nobreak{}
    \stepcounter{#1}
    \nobreak\extramarks{Problem \arabic{#1}}{}\nobreak{}
}

\setcounter{secnumdepth}{0}
\newcounter{partCounter}
\newcounter{homeworkProblemCounter}
\setcounter{homeworkProblemCounter}{1}
\nobreak\extramarks{Problem \arabic{homeworkProblemCounter}}{}\nobreak{}

%
% Homework Problem Environment
%
% This environment takes an optional argument. When given, it will adjust the
% problem counter. This is useful for when the problems given for your
% assignment aren't sequential. See the last 3 problems of this template for an
% example.
%
\newenvironment{homeworkProblem}[1][-1]{
    \ifnum#1>0
        \setcounter{homeworkProblemCounter}{#1}
    \fi
    \section{Problem \arabic{homeworkProblemCounter}}
    \setcounter{partCounter}{1}
    \enterProblemHeader{homeworkProblemCounter}
}{
    \exitProblemHeader{homeworkProblemCounter}
}

%
% Homework Details
%   - Title
%   - Due date
%   - Class
%   - Section/Time
%   - Instructor
%   - Author
%

\newcommand{\hmwkTitle}{Homework\ \#2}
\newcommand{\hmwkDueDate}{January 19, 2015}
\newcommand{\hmwkClass}{Numerical Analysis}
\newcommand{\hmwkClassTime}{MWF 2:15}
\newcommand{\hmwkClassInstructor}{Professor Mohler}
\newcommand{\hmwkAuthorName}{Rick Sullivan}

%
% Title Page
%

\title{
    \vspace{2in}
    \textmd{\textbf{\hmwkClass:\ \hmwkTitle}}\\
    \normalsize\vspace{0.1in}\small{Due\ on\ \hmwkDueDate}\\
    \vspace{0.1in}\large{\textit{\hmwkClassInstructor\ \hmwkClassTime}}
    \vspace{3in}
}

\author{\textbf{\hmwkAuthorName}}
\date{}

\renewcommand{\part}[1]{\textbf{\large Part \Alph{partCounter}}\stepcounter{partCounter}\\}

%
% Various Helper Commands
%

% Useful for algorithms
\newcommand{\alg}[1]{\textsc{\bfseries \footnotesize #1}}

% For derivatives
\newcommand{\deriv}[1]{\frac{\mathrm{d}}{\mathrm{d}x} (#1)}

% For partial derivatives
\newcommand{\pderiv}[2]{\frac{\partial}{\partial #1} (#2)}

% Integral dx
\newcommand{\dx}{\mathrm{d}x}

% Alias for the Solution section header
\newcommand{\solution}{\textbf{\large Solution}}

% Probability commands: Expectation, Variance, Covariance, Bias
\newcommand{\E}{\mathrm{E}}
\newcommand{\Var}{\mathrm{Var}}
\newcommand{\Cov}{\mathrm{Cov}}
\newcommand{\Bias}{\mathrm{Bias}}

\begin{document}

\maketitle

\pagebreak

\begin{homeworkProblem}
    Let \(f(x) = -x^3 - \cos{x}\) and \(p_0=1\). Use two iterations of Newtown's method to approximate the root of \(f(x)\).
\\

    \textbf{Solution}

    Iteration one:
    \[
        \begin{split}
            f'(x) &= -3x^2 + \sin{x}
        \end{split}
    \]

    Use the slope at \(x = p_0 = 1\) to approximate the root for our next iteration.

    \[
        \begin{split}
            f(1) &\approx -1.5403 \\
            f'(1) &\approx -2.15853 
            \\
            p_1 &= x_0 - \frac{f(x_0)}{f'(x_0)}\\
            &= 1 - \frac{-1.5403}{-2.15853}\\
            &= 0.2864
        \end{split}
    \]

    Iteration two with \(p_1 = 0.2864\):

    \[
        \begin{split}
            f(0.2864) &\approx -0.982759 \\
            f'(0.2864) &\approx 0.0364182
            \\
            p_2 &= x_1 - \frac{f(x_1)}{f'(x_1)}\\
            &= 0.2864 - \frac{-0.982759}{0.0364182}\\
            &= 27.98538
        \end{split}
    \]

    Our final approximation is 27.9854. This is very innacurate because the slope at our initial choice is nearly 0, pushing our estimation far from the actual root.
\end{homeworkProblem}

\begin{homeworkProblem}
    The derivative of a function can be approximated by 
    \[
        \begin{split}
            f'(x) &\approx \frac{-3f(x) + 4f(x+h) - f(x+2h)}{2h}
        \end{split}
    \]

    Show that the error of this approximation is \(O(h^2)\).
\\

    \textbf{Solution}

    The Taylor series for \(f(x + h)\) and \(f(x + 2h)\) are as follows.
    \[
        \begin{split}
            f(x + h) &= f(x) + hf'(x) + \frac{h^2}{2}f''(x) + \frac{h^3}{6}f'''(x) + Ch^4
            \\
            f(x + 2h) &= f(x) + 2hf'(x) + 2h^2f''(x) + (4/3)h^3f'''(x) + Ch^4
        \end{split}
    \]

    Using those expansions to calculate the error of the approximation
 
    \[
        \begin{split}
            err_h &= f'(x) - \frac{-3f(x) + 4f(x + h) - f(x + 2h)}{2h}
           \\
           err_h &= f'(x) - \frac{-3f(x) + 4\left[f(x) + hf'(x) + \frac{h^2}{2}f''(x) + \frac{h^3}{6}f'''(x) + Ch^4\right]}{2h} 
           \\
           &\hspace{0.5in} - \frac{\left[f(x) + 2hf'(x) + 2h^2f''(x) + (4/3)h^3f'''(x) + Ch^4\right]}{2h}
           \\
            &= f'(x) - \frac{4hf'(x) + 2h^2f''(x) + \frac{2h^3}{3}f'''(x) - 2hf'(x) - 2h^2f''(x) - \frac{4h^3}{3}f'''(x) + Ch^4}{2h}
            \\
            &= f'(x) - \frac{2hf'(x) - \frac{2h^3}{3}f'''(x) + Ch^4}{2h}
            \\
            &= f'(x) - f'(x) - \frac{h^2}{3}f'''(x) + Ch^3
            \\
            &= \frac{h^2}{3}f'''(x) + Ch^3
        \end{split}
    \]

    If h is small, the first term will be much more significant than any other term in the expansion. Therefore, the error of this approximation is \(O(h^2)\).
\end{homeworkProblem}

\begin{homeworkProblem}
    Use the approximation for \(f''(x)\) given in class to approximate the second derivative of
    \[
        \begin{split}
            f(x) &= \cos{e^{x^2} + x^3}
        \end{split}
    \]

    at \(x = 1\) using \(h = .1,.05,.025\). How does the decrease in error compare to the order of accuracy of the method (is the decrease what we should expect)?
\\

    \textbf{Solution}

    The equation for \(f''(x)\) given in class:
    \[
        \begin{split}
            f''(x) &= \frac{f(x+h) - 2f(x) + f(x-h)}{h^2}
        \end{split}
    \]

    Using the specified inputs:
    \[
        \begin{split}
            f''(1)_{0.1} &= \frac{f(1+0.1) - 2f(1) + f(1-0.1)}{0.1^2}
            \\
            &\approx 0.3534
            \\
            f''(1)_{0.05} &= \frac{f(1+0.05) - 2f(1) + f(1-0.05)}{0.05^2}
            \\
            &\approx 0.1661
            \\
            f''(1)_{0.025} &= \frac{f(1+0.025) - 2f(1) + f(1-0.025)}{0.025^2}
            \\
            &\approx 0.1164
        \end{split}
    \]

In class, we determined that the calculated errror of this approach is \(O(h^2)]\). Therefore, estimation with choice \(h_1\) should have 4 times the error of an estimation with choice \(h_2 = (1/2)h_1\).

    \[
        \begin{split}
            \frac{f''(1)_{0.1}}{f''(1)_{0.05}} = 2.1276
            \\
            \frac{f''(1)_{0.05}}{f''(1)_{0.025}} = 1.4270
        \end{split}
    \]

ADD SOMETHING HERE. ARE THESE RESULTS CORRECT?

\end{homeworkProblem}

\begin{homeworkProblem}
    Use the composite trapezoidal rule and Simpson's rule with \(n = 4\) to approximate
    \[
        \begin{split}
            \int_0^1 x^2e^{-x}
        \end{split}
    \]

    What is the error of each method?
\\

    \textbf{Solution}

    With \(n=4\), we will divide the interval \([0, 1]\) into 4 segments with \(h=0.25\). Using the composite trapezoidal rule:

    \[
        \begin{split}
            \int_0^1 x^2e^{-x} &= \sum\limits_{i=0}^3 \frac{1}{4} \frac{f(x_{i}) + f(x_{i+1})}{2}
        \\
        &= \frac{1}{8} \left[f(0) + 2f(1/4) + 2f(1/2) + 2f(3/4) + f(1)\right]
        \\
        &= \frac{1}{8} \left[1 + \frac{1}{8e^{1/4}} + \frac{1}{2e^{1/2}} + \frac{9}{8e^{3/4}} + \frac{1}{e} \right]
        \\
        &\approx 0.2848
        \end{split}
    \]


    Using the composite Simpson's rule:

    \[
        \begin{split}
            \int_0^1 x^2e^{-x} &= \sum\limits_{i=0}^1 \frac{0.25}{3} \left[f(x_{2i}) + 4f(x_{2i+1}) + f(x_{2i+2})\right]
        \\
        &= \frac{1}{12} \left[(1 + \frac{1}{4e^{1/4}} + \frac{1}{4e^{1/2}}) + (\frac{1}{4e^{1/2}} + \frac{9}{4e^{3/4}} + \frac{1}{e})\right] 
        \\
        &\approx 0.2441
        \end{split}
    \]

    TODO: ENTER ERROR OF THIS METHOD
    % Wolfram alpha answer: ~~ 0.16060
\end{homeworkProblem}

\begin{homeworkProblem}
    Determine \(c_0\), \(c_1\), and \(c_2\) such that 
    \[
        \begin{split}
            \int_0^2 p(x) dx &= c_0p(0) + c_1p(1) + c_2p(2)
        \end{split}
    \]

    for all polynomials \(p(x)\) of degree 2 or less.
\\

    \textbf{Solution}

    Simpson's rule provides exact estimation for polynomials up to degree 3.

    \[
        \begin{split}
            \int_0^2 p(x) dx &\approx \frac{1}{3} \left[p(0) + 4p(1) + p(2)\right]
            \\
        &= \frac{1}{3}p(0) + \frac{4}{3}p(1) + \frac{1}{3}p(2)
            \\
        \end{split}
    \]

    Therefore, \(c_0=1/3\), \(c_1=4/3\), and \(c_2=1/3\).


\end{homeworkProblem}
%\pagebreak
%\section{Appendix}

\end{document}
