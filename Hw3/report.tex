\documentclass{article}

\usepackage{fancyhdr}
\usepackage{extramarks}
\usepackage{amsmath}
\usepackage{amsthm}
\usepackage{amsfonts}
\usepackage{tikz}
\usepackage[plain]{algorithm}
\usepackage{algpseudocode}
\usepackage{listings}
\usepackage{enumerate}
\lstset{breaklines=true}

\usetikzlibrary{automata,positioning}

%
% Basic Document Settings
%

\topmargin=-0.45in
\evensidemargin=0in
\oddsidemargin=0in
\textwidth=6.5in
\textheight=9.0in
\headsep=0.25in

\linespread{1.1}

\pagestyle{fancy}
\lhead{\hmwkAuthorName}
\chead{\hmwkClass\ (\hmwkClassInstructor): \hmwkTitle}
\rhead{\firstxmark}
\lfoot{\lastxmark}
\cfoot{\thepage}

\renewcommand\headrulewidth{0.4pt}
\renewcommand\footrulewidth{0.4pt}

\setlength\parindent{0pt}

%
% Create Problem Sections
%

\newcommand{\enterProblemHeader}[1]{
    \nobreak\extramarks{}{Problem \arabic{#1} continued on next page\ldots}\nobreak{}
    \nobreak\extramarks{Problem \arabic{#1} (continued)}{Problem \arabic{#1} continued on next page\ldots}\nobreak{}
}

\newcommand{\exitProblemHeader}[1]{
    \nobreak\extramarks{Problem \arabic{#1} (continued)}{Problem \arabic{#1} continued on next page\ldots}\nobreak{}
    \stepcounter{#1}
    \nobreak\extramarks{Problem \arabic{#1}}{}\nobreak{}
}

\setcounter{secnumdepth}{0}
\newcounter{partCounter}
\newcounter{homeworkProblemCounter}
\setcounter{homeworkProblemCounter}{1}
\nobreak\extramarks{Problem \arabic{homeworkProblemCounter}}{}\nobreak{}

%
% Homework Problem Environment
%
% This environment takes an optional argument. When given, it will adjust the
% problem counter. This is useful for when the problems given for your
% assignment aren't sequential. See the last 3 problems of this template for an
% example.
%
\newenvironment{homeworkProblem}[1][-1]{
    \ifnum#1>0
        \setcounter{homeworkProblemCounter}{#1}
    \fi
    \section{Problem \arabic{homeworkProblemCounter}}
    \setcounter{partCounter}{1}
    \enterProblemHeader{homeworkProblemCounter}
}{
    \exitProblemHeader{homeworkProblemCounter}
}

%
% Homework Details
%   - Title
%   - Due date
%   - Class
%   - Section/Time
%   - Instructor
%   - Author
%

\newcommand{\hmwkTitle}{Homework\ \#3}
\newcommand{\hmwkDueDate}{February 2, 2015}
\newcommand{\hmwkClass}{Numerical Analysis}
\newcommand{\hmwkClassTime}{MWF 2:15}
\newcommand{\hmwkClassInstructor}{Professor Mohler}
\newcommand{\hmwkAuthorName}{Rick Sullivan}

%
% Title Page
%

\title{
    \vspace{2in}
    \textmd{\textbf{\hmwkClass:\ \hmwkTitle}}\\
    \normalsize\vspace{0.1in}\small{Due\ on\ \hmwkDueDate}\\
    \vspace{0.1in}\large{\textit{\hmwkClassInstructor\ \hmwkClassTime}}
    \vspace{3in}
}

\author{\textbf{\hmwkAuthorName}}
\date{}

\renewcommand{\part}[1]{\textbf{\large Part \Alph{partCounter}}\stepcounter{partCounter}\\}

%
% Various Helper Commands
%

% Useful for algorithms
\newcommand{\alg}[1]{\textsc{\bfseries \footnotesize #1}}

% For derivatives
\newcommand{\deriv}[1]{\frac{\mathrm{d}}{\mathrm{d}x} (#1)}

% For partial derivatives
\newcommand{\pderiv}[2]{\frac{\partial}{\partial #1} (#2)}

% Integral dx
\newcommand{\dx}{\mathrm{d}x}

% Alias for the Solution section header
\newcommand{\solution}{\textbf{\large Solution}}

% Probability commands: Expectation, Variance, Covariance, Bias
\newcommand{\E}{\mathrm{E}}
\newcommand{\Var}{\mathrm{Var}}
\newcommand{\Cov}{\mathrm{Cov}}
\newcommand{\Bias}{\mathrm{Bias}}

\begin{document}

\maketitle

\pagebreak

\begin{homeworkProblem}
    Use Modified Euler's method to approximate the solution to

    \[
        \begin{split}
            y' &= \cos(yt), y(0) = 3
        \end{split}
    \]

    on the interval \([0,2]\) with \(h=.5\).
    \\

    \textbf{Solution}

    \[
        \begin{split}
            \frac{w_{i+1}-w_i}{h} &= \frac{f(t_{i+1}, \widetilde{w}_{i+1}) + f(t_i, w_i)}{2}
            \\
            f(t_{i+1}, \widetilde{w}_{i+1}) &= \frac{2(w_{i+1}-w_i)}{h} - f(t_i, w_i)
            \\
            \widetilde{w}_{i+1} &= w_i + hf(t_i, w_i)
            \\
            w_1 &= 3 + .5(\cos(0*3))
            \\
            &= 3 + .5
            \\
            &= 3.5
            \\
            y(.5) &= \frac{2(3.5-3)}{.5} - y(0)
            \\
            &= 2 - 3
            \\
            &= -1
            \\
            w_2 &= 3.5 + .5(-1)
            \\
            &= 3
            \\
            y(1) &= \frac{2(3-3.5)}{.5} - (-1)
            \\
            &= -1 + 1
            \\
            &= 0
            \\
            w_3 &= 3 + .5(0)
            \\
            &= 3
            \\
            y(1.5) &= \frac{2(3-3)}{.5} - 0
            \\
            &= 0
            \\
            w_4 &= 3 - .5(0)
            \\
            &= 3
            \\
            y(2) &= \frac{2(3-3)}{.5} - 0
            \\
            &= 0
        \end{split}
    \]

    The approximation of the solution converges to 0 on the given interval.
\end{homeworkProblem}

\pagebreak

\begin{homeworkProblem}
    Consider the initial value problem

    \[
        \begin{split}
            y' &= -200y + \cos(t), y(0) = 1
        \end{split}
    \]

    \begin{enumerate}[a)]
        \item What happens to Forward Euler's method if \(h=.1\) is used? Compute a few iterations and discuss the results.

        \item What condition is needed on \(h\) for Forward Euler's method to give reasonable results?

        \item What is a method that doesn't have this restriction on \(h\)? Use this method to approximate \(y(.2)\).
    \end{enumerate}

    \textbf{Part a}
    \\

    \[
        \begin{split}
            y_{i+1} &= y_i + hy'(t_i, y_i)
            \\
            y(.1) &= y(0) + .1y'(0, 1) \\
                  &= 1 + .1(-199) \\
                  &= -18.9 \\
            y(.2) &= y(.1) + .1f(.1, -18.9) \\
                  &= -18.9 + .1(3781) \\
                  &= 359.2 \\
            y(.3) &= y(.2) + .1f(.2, 359.2) \\
                  &= 359.2 + .1(-71839) \\
                  &= -6824.7
        \end{split}
    \]

    The estimations are diverging and oscillating.
    \\

    \textbf{Part b}
    \\

    \textbf{Part c}
    \\

    Runga-Kutta methods don't have this stability problem.
    \[
        \begin{split}
            k_1 &= .1y'(0, 1)\\
                &= .1(-199) \\
                &= -19.9 \\
            k_2 &= .1y'(.05, 1 + k_1/2) \\
              &= .1y'(.05, -8.95) \\
              &= .1(1791) \\
              &= 179.1 \\
            k_3 &= .1y'(.05, 1 + k_2/2) \\
                &= .1y'(.05, 89.65) \\
                &= .1(-17929) \\
                &= -1792.9\\
            k_4 &= .1y'(.1, 1 + k_3) \\
                &= .1y'(.1, -1791.9) \\
                &= 35857.8 \\
            y(.1) &= 1 + \frac{k_1 + k_2 + k_3 + k_4}{6}\\
        &= 5680.12167
        \end{split}
    \]

    Trying Modified Euler:

    \[
        \begin{split}
            w_0 &= 0
            \\
            w_1 &= 0 + .1(1)
            \\
            &= .1
            \\
            y(.1) &= \frac{2(.1-0)}{.1} - 1
            \\
            &= 1
            \\
            w_2 &= .1 + .1(1)
            \\
            &= .2
            \\
            y(.2) &= \frac{2(.2-1)}{.1} - 1
            \\
            &= 1
        \end{split}
    \]
\end{homeworkProblem}

\begin{homeworkProblem}
    Lotka-Volterra equations are used to model predator-prey systems. For example

    \[
        \begin{split}
            c' &= .01cr - .5c
        \end{split}
    \]
    and
    \[
        \begin{split}
            r' &= 2r - .02cr
        \end{split}
    \]

    might model a population of coyotes and rabbits. Use Forward Euler and \(h=.5\) to estimate the populations at time \(t=1\) if \(c(0)=r(0)=100\).
    \\

    \textbf{Solution}
    \\

    \[
        \begin{split}
            c(.5) &= 100 + .5(.01(100)(100) - .5(100))
            \\
            &= 100 + .5(50)
            \\
            &= 125
            \\
            r(.5) &= 100 + .5(2(100) - .02(100)(100))
            \\
            &= 100 + .5(200 - 200)
            \\
            &= 100
            \\
            c(1) &= c_{.5} + .5(.01c_{.5}r_{.5} - .5c_{.5})
            \\
            &= 125 + .5(.01(125)(100) - .5(125))
            \\
            &= 125 + .5(125 - 62.5)
            \\
            &= 125 + 31.25
            \\
            &= 156.25
            \\
            r(1) &= r_{.5} + .5(2r_{.5} - .02r_{.5}c_{.5})
            \\
            &= 100 + .5(2(100) - .02(100)(125))
            \\
            &= 100 + .5(200 - 250)
            \\
            &= 75
        \end{split}
    \]

    Our final estimations using Forward Euler have \(c(1) = 156.25\) and \(r(1) = 75\).
\end{homeworkProblem}
%\pagebreak
%\section{Appendix}

\end{document}
