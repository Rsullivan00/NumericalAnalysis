\documentclass{article}

\usepackage{fancyhdr}
\usepackage{extramarks}
\usepackage{amsmath}
\usepackage{amsthm}
\usepackage{amsfonts}
\usepackage{tikz}
\usepackage[plain]{algorithm}
\usepackage{algpseudocode}
\usepackage{listings}
\usepackage{enumerate}
\lstset{breaklines=true}

\usetikzlibrary{automata,positioning}

%
% Basic Document Settings
%

\topmargin=-0.45in
\evensidemargin=0in
\oddsidemargin=0in
\textwidth=6.5in
\textheight=9.0in
\headsep=0.25in

\linespread{1.1}

\pagestyle{fancy}
\lhead{\hmwkAuthorName}
\chead{\hmwkClass\ (\hmwkClassInstructor): \hmwkTitle}
\rhead{\firstxmark}
\lfoot{\lastxmark}
\cfoot{\thepage}

\renewcommand\headrulewidth{0.4pt}
\renewcommand\footrulewidth{0.4pt}

\setlength\parindent{0pt}

%
% Create Problem Sections
%

\newcommand{\enterProblemHeader}[1]{
    \nobreak\extramarks{}{Problem \arabic{#1} continued on next page\ldots}\nobreak{}
    \nobreak\extramarks{Problem \arabic{#1} (continued)}{Problem \arabic{#1} continued on next page\ldots}\nobreak{}
}

\newcommand{\exitProblemHeader}[1]{
    \nobreak\extramarks{Problem \arabic{#1} (continued)}{Problem \arabic{#1} continued on next page\ldots}\nobreak{}
    \stepcounter{#1}
    \nobreak\extramarks{Problem \arabic{#1}}{}\nobreak{}
}

\setcounter{secnumdepth}{0}
\newcounter{partCounter}
\newcounter{homeworkProblemCounter}
\setcounter{homeworkProblemCounter}{1}
\nobreak\extramarks{Problem \arabic{homeworkProblemCounter}}{}\nobreak{}

%
% Homework Problem Environment
%
% This environment takes an optional argument. When given, it will adjust the
% problem counter. This is useful for when the problems given for your
% assignment aren't sequential. See the last 3 problems of this template for an
% example.
%
\newenvironment{homeworkProblem}[1][-1]{
    \ifnum#1>0
        \setcounter{homeworkProblemCounter}{#1}
    \fi
    \section{Problem \arabic{homeworkProblemCounter}}
    \setcounter{partCounter}{1}
    \enterProblemHeader{homeworkProblemCounter}
}{
    \exitProblemHeader{homeworkProblemCounter}
}

%
% Homework Details
%   - Title
%   - Due date
%   - Class
%   - Section/Time
%   - Instructor
%   - Author
%

\newcommand{\hmwkTitle}{Homework\ \#4}
\newcommand{\hmwkDueDate}{February 11, 2015}
\newcommand{\hmwkClass}{Numerical Analysis}
\newcommand{\hmwkClassTime}{MWF 2:15}
\newcommand{\hmwkClassInstructor}{Professor Mohler}
\newcommand{\hmwkAuthorName}{Rick Sullivan}

%
% Title Page
%

\title{
    \vspace{2in}
    \textmd{\textbf{\hmwkClass:\ \hmwkTitle}}\\
    \normalsize\vspace{0.1in}\small{Due\ on\ \hmwkDueDate}\\
    \vspace{0.1in}\large{\textit{\hmwkClassInstructor\ \hmwkClassTime}}
    \vspace{3in}
}

\author{\textbf{\hmwkAuthorName}}
\date{}

\renewcommand{\part}[1]{\textbf{\large Part \Alph{partCounter}}\stepcounter{partCounter}\\}

%
% Various Helper Commands
%

% Useful for algorithms
\newcommand{\alg}[1]{\textsc{\bfseries \footnotesize #1}}

% For derivatives
\newcommand{\deriv}[1]{\frac{\mathrm{d}}{\mathrm{d}x} (#1)}

% For partial derivatives
\newcommand{\pderiv}[2]{\frac{\partial}{\partial #1} (#2)}

% Integral dx
\newcommand{\dx}{\mathrm{d}x}

% Alias for the Solution section header
\newcommand{\solution}{\textbf{\large Solution}}

% Probability commands: Expectation, Variance, Covariance, Bias
\newcommand{\E}{\mathrm{E}}
\newcommand{\Var}{\mathrm{Var}}
\newcommand{\Cov}{\mathrm{Cov}}
\newcommand{\Bias}{\mathrm{Bias}}

\begin{document}

\maketitle

\pagebreak

\begin{homeworkProblem}
    \begin{enumerate}[a)]
        \item Use the Gaussian elimination algorithm discussed in class to put the following augmented matrix into triangular (reduced) form.

            \[
                \begin{split}
                    A &=
                    \begin{bmatrix}
                        1 & -1 & 3 & 2\\
                        3 & -3 & 1 & -1\\
                        1 & 1 & 0 & 3\\
                    \end{bmatrix}
                \end{split}
            \]

        \item Use backward substitution to solve the system.
    \end{enumerate}


    \textbf{Part a}
    \\
         \[
            \begin{split}
                A &=
                \begin{bmatrix}
                    1 & -1 & 3 & 2\\
                    3 & -3 & 1 & -1\\
                    1 & 1 & 0 & 3\\
                \end{bmatrix}
                \\
                &=
                \begin{bmatrix}
                    1 & -1 & 3 & 2\\
                    1 & -1 & 1/3 & -1/3\\
                    1 & 1 & 0 & 3\\
                \end{bmatrix}
                \\
                &=
                \begin{bmatrix}
                    1 & -1 & 3 & 2\\
                    0 & 0 & 8/3 & 7/3\\
                    1 & 1 & 0 & 3\\
                \end{bmatrix}
                \\
                &=
                \begin{bmatrix}
                    1 & -1 & 3 & 2\\
                    0 & 0 & 1 & 7/8\\
                    1 & 1 & 0 & 3\\
                \end{bmatrix}
                \\
                &=
                \begin{bmatrix}
                    1 & -1 & 3 & 2\\
                    1 & 1 & 0 & 3\\
                    0 & 0 & 1 & 7/8\\
                \end{bmatrix}
                \\
                &=
                \begin{bmatrix}
                    1 & -1 & 3 & 2\\
                    0 & 2 & -3 & 1\\
                    0 & 0 & 1 & 7/8\\
                \end{bmatrix}
                \\
                &=
                \begin{bmatrix}
                    1 & -1 & 3 & 2\\
                    0 & 1 & -3/2 & 1/2\\
                    0 & 0 & 1 & 7/8\\
                \end{bmatrix}
            \end{split}
        \]
   
    \textbf{Part b}
    \\

    \[
        \begin{split}
            A &=
            \begin{bmatrix}
                1 & -1 & 3 & 2\\
                0 & 1 & -3/2 & 1/2\\
                0 & 0 & 1 & 7/8\\
            \end{bmatrix}
            \\
            &=
             \begin{bmatrix}
                1 & -1 & 3 & 2\\
                0 & 1 & 0 & 29/16\\
                0 & 0 & 1 & 7/8\\
            \end{bmatrix}
            \\
            &=
            \begin{bmatrix}
                1 & 0 & 0 & 19/16\\
                0 & 1 & 0 & 29/16\\
                0 & 0 & 1 & 7/8\\
            \end{bmatrix}
        \end{split}
    \]
\end{homeworkProblem}

\begin{homeworkProblem}
    \begin{enumerate}[a)]
        \item Factor the matrix
            \[
                \begin{split}
                    A &=
                    \begin{bmatrix}
                        4 & 0 & 0\\
                        2 & 8 & 2\\
                        4 & 4 & 16\\
                    \end{bmatrix}
                \end{split}
            \]
            into the product of a lower and upper triangular matrix.

        \item Solve the linear system \(Ax = b\), where \(b\) is the vector of all 1's, using \(A = LU\) from part a) and forward/backward substitution.
    \end{enumerate}

    \textbf{Part a}
    \\
        \[
            \begin{split}
                A &=
                \begin{bmatrix}
                    4 & 0 & 0\\
                    2 & 8 & 2\\
                    4 & 4 & 16\\
                \end{bmatrix}
                = 
                \begin{bmatrix}
                    l_{1,1} & 0 & 0\\
                    l_{2,1} & l_{2,2} & 0\\
                    0 & l_{3,2} & l_{3,3}\\
                \end{bmatrix}
                \begin{bmatrix}
                    1 & u_{1,2} & 0\\
                    0 & 1 & u_{2,3}\\
                    0 & 0 & 1\\
                \end{bmatrix}
                \\
                l_{1,1} &= 4
                \\
                2 &= l_{2,1} * 1
                \\
                l_{2,1} &= 2
                \\
                0 &= 4 * u_{1,2}
                \\
                u_{1,2} &= 0
            \end{split}
        \]

        So we now have 
        \[
            \begin{split}
                A &=
                \begin{bmatrix}
                    4 & 0 & 0\\
                    2 & l_{2,2} & 0\\
                    0 & l_{3,2} & l_{3,3}\\
                \end{bmatrix}
                \begin{bmatrix}
                    1 & 0 & 0\\
                    0 & 1 & u_{2,3}\\
                    0 & 0 & 1\\
                \end{bmatrix}
                \\
                8 &= 2 * 0 + l_{2,2} * 1
                \\
                l_{2,2} &= 8
                \\
                2 &= 8 * u_{2,3}
                \\
                u_{2,3} &= 1/4
                \\
                4 &= 1 * l_{3,2} 
                \\
                l_{3,2} &= 4
                \\
                16 &= 4 * (1/4) + 1 * l_{3,3}
                \\
                l_{3,3} &= 15
            \end{split}
        \]

        Finally,
        \[
            \begin{split}
                A &=
                \begin{bmatrix}
                    4 & 0 & 0\\
                    2 & 8 & 0\\
                    0 & 4 & 15\\
                \end{bmatrix}
                \begin{bmatrix}
                    1 & 0 & 0\\
                    0 & 1 & 1/4\\
                    0 & 0 & 1\\
                \end{bmatrix}
            \end{split}
        \]


    \textbf{Part b}
    \\

    \[
        \begin{split}
            Ax = LUx &= b 
            \\
        \end{split}
    \]
    First solving for \(y\) in \(Ly=b\).
    \[
        \begin{split}
            \begin{bmatrix}
                4 & 0 & 0\\
                2 & 8 & 0\\
                0 & 4 & 15\\
            \end{bmatrix}
            \begin{bmatrix}
                y_1\\
                y_2\\
                y_3\\
            \end{bmatrix}      
            &= 
            \begin{bmatrix}
                1\\
                1\\
                1\\
            \end{bmatrix}      
            \\
            1 &= 4y_1
            \\
            y_1 &= 1/4
            \\
            1 &= 2(1/4) + 8y_2
            \\
            y_2 &= 1/16
            \\
            1 &= 4(1/16) + 15y_3
            \\
            y_3 &= 1/20
        \end{split}
    \]
    Now solving for \(x\) in \(Ux=y\).
    \[
        \begin{split}
            \begin{bmatrix}
                1 & 0 & 0\\
                0 & 1 & 1/4\\
                0 & 0 & 1\\
            \end{bmatrix}
            \begin{bmatrix}
                x_1\\
                x_2\\
                x_3\\
            \end{bmatrix}      
            &=
            \begin{bmatrix}
                1/4\\
                1/16\\
                1/20\\
            \end{bmatrix}      
            \\
            x_1 &= 1/4
            \\
            x_3 &= 1/20
            \\
            x_2 &= 1/16 - 1/4(1/20)
            \\
            &= 1/20
            \\
            x &= 
            \begin{bmatrix}
                1/4\\
                1/16\\
                1/20\\
            \end{bmatrix}
        \end{split}
    \]
\end{homeworkProblem}

\begin{homeworkProblem}
        Let
            \[
                \begin{split}
                    A &=
                    \begin{bmatrix}
                        \alpha & 1 & 0\\
                        \beta & 2 & 1\\
                        0 & 1 & 2\\
                    \end{bmatrix}
                \end{split}
            \]
        Find all values of \(\alpha\) and \(\beta\) such that 
        \begin{enumerate}[a)]
            \item A is singular (non-invertible)
            \item A is symmetric
            \item A is strictly diagonally dominant
            \item A is positive definite
        \end{enumerate}

        \textbf{Solution}
        \\

        \begin{enumerate}
            \item A singular matrix has a determinant of 0.
                \[
                    \begin{split}
                        |A| &= 0
                        \\
                        (\alpha*2*2) - (1*1*\alpha + 2*1*\beta) &= 0
                        \\
                        4\alpha - \alpha - 2\beta &= 0
                        \\
                        \alpha &= (2/3)\beta
                    \end{split}
                \]
                A is singular for all \(\alpha\) and \(\beta\) where \(\alpha = (2/3)\beta\).

            \item A is symmetric for \(\alpha = 2\) and \(\beta = 1\).

            \item A is strictly diagonally dominant if its determinant is nonzero. From a), A is strictly diagonally dominant when \(\alpha \neq (2/3)\beta\).

            \item Whaaa?
        \end{enumerate}
\end{homeworkProblem}

\begin{homeworkProblem}
    \begin{enumerate}[a)]
        \item Find \(||x||_\infty\) and \(||x||_2\) for \(x=[\cos(k),\sin(k),e^k]^T\) and a \(k\) a positive integer.

        \item Find \(||A||_\infty\) for
            \[
                \begin{split}
                    A &=
                    \begin{bmatrix}
                        2 & -1 & 0\\
                        -1 & 2 & -1\\
                        0 & -1 & 2\\
                    \end{bmatrix}
                \end{split}
            \]
    \end{enumerate}

    \textbf{Part a}
    \\

    \textbf{Part b}
\end{homeworkProblem}


%\pagebreak
%\section{Appendix}

\end{document}
