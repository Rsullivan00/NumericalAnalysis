\documentclass{article}

\usepackage{fancyhdr}
\usepackage{extramarks}
\usepackage{amsmath}
\usepackage{amsthm}
\usepackage{amsfonts}
\usepackage{tikz}
\usepackage[plain]{algorithm}
\usepackage{algpseudocode}
\usepackage{listings}
\usepackage{enumerate}
\lstset{breaklines=true}

\usetikzlibrary{automata,positioning}

%
% Basic Document Settings
%

\topmargin=-0.45in
\evensidemargin=0in
\oddsidemargin=0in
\textwidth=6.5in
\textheight=9.0in
\headsep=0.25in

\linespread{1.1}

\pagestyle{fancy}
\lhead{\hmwkAuthorName}
\chead{\hmwkClass\ (\hmwkClassInstructor): \hmwkTitle}
\rhead{\firstxmark}
\lfoot{\lastxmark}
\cfoot{\thepage}

\renewcommand\headrulewidth{0.4pt}
\renewcommand\footrulewidth{0.4pt}

\setlength\parindent{0pt}

%
% Create Problem Sections
%

\newcommand{\enterProblemHeader}[1]{
    \nobreak\extramarks{}{Problem \arabic{#1} continued on next page\ldots}\nobreak{}
    \nobreak\extramarks{Problem \arabic{#1} (continued)}{Problem \arabic{#1} continued on next page\ldots}\nobreak{}
}

\newcommand{\exitProblemHeader}[1]{
    \nobreak\extramarks{Problem \arabic{#1} (continued)}{Problem \arabic{#1} continued on next page\ldots}\nobreak{}
    \stepcounter{#1}
    \nobreak\extramarks{Problem \arabic{#1}}{}\nobreak{}
}

\setcounter{secnumdepth}{0}
\newcounter{partCounter}
\newcounter{homeworkProblemCounter}
\setcounter{homeworkProblemCounter}{1}
\nobreak\extramarks{Problem \arabic{homeworkProblemCounter}}{}\nobreak{}

%
% Homework Problem Environment
%
% This environment takes an optional argument. When given, it will adjust the
% problem counter. This is useful for when the problems given for your
% assignment aren't sequential. See the last 3 problems of this template for an
% example.
%
\newenvironment{homeworkProblem}[1][-1]{
    \ifnum#1>0
        \setcounter{homeworkProblemCounter}{#1}
    \fi
    \section{Problem \arabic{homeworkProblemCounter}}
    \setcounter{partCounter}{1}
    \enterProblemHeader{homeworkProblemCounter}
}{
    \exitProblemHeader{homeworkProblemCounter}
}

%
% Homework Details
%   - Title
%   - Due date
%   - Class
%   - Section/Time
%   - Instructor
%   - Author
%

\newcommand{\hmwkTitle}{Homework\ \#7}
\newcommand{\hmwkDueDate}{May 4, 2015}
\newcommand{\hmwkClass}{Numerical Analysis}
\newcommand{\hmwkClassTime}{MWF 2:15}
\newcommand{\hmwkClassInstructor}{Professor Mohler}
\newcommand{\hmwkAuthorName}{Rick Sullivan}

%
% Title Page
%

\title{
    \vspace{2in}
    \textmd{\textbf{\hmwkClass:\ \hmwkTitle}}\\
    \normalsize\vspace{0.1in}\small{Due\ on\ \hmwkDueDate}\\
    \vspace{0.1in}\large{\textit{\hmwkClassInstructor\ \hmwkClassTime}}
    \vspace{3in}
}

\author{\textbf{\hmwkAuthorName}}
\date{}

\renewcommand{\part}[1]{\textbf{\large Part \Alph{partCounter}}\stepcounter{partCounter}\\}

%
% Various Helper Commands
%

% Useful for algorithms
\newcommand{\alg}[1]{\textsc{\bfseries \footnotesize #1}}

% For derivatives
\newcommand{\deriv}[1]{\frac{\mathrm{d}}{\mathrm{d}x} (#1)}

% For partial derivatives
\newcommand{\pderiv}[2]{\frac{\partial}{\partial #1} (#2)}

% Integral dx
\newcommand{\dx}{\mathrm{d}x}

% Alias for the Solution section header
\newcommand{\solution}{\textbf{\large Solution}}

% Probability commands: Expectation, Variance, Covariance, Bias
\newcommand{\E}{\mathrm{E}}
\newcommand{\Var}{\mathrm{Var}}
\newcommand{\Cov}{\mathrm{Cov}}
\newcommand{\Bias}{\mathrm{Bias}}

\begin{document}

\maketitle

\pagebreak

\begin{homeworkProblem}
    Consider the nonlinear system
    \[
        \begin{split}
            5x_1^2 - x_2^2 &= 0\\
            x_2 - (\sin(x_1) + \cos(x_2))/4 &= 0
        \end{split}
    \]

    \begin{enumerate}[a)]
        \item Find a function \(G(\vec{x})\) and a set \(D\) in \(\mathbb{R}^2\) such that \(G\) has a unique fixed point in \(D\).
        \item Estimate the number of iterations required to approximate the exact solution within \(10^{-5}\) in the \(||\cdot||_\infty\) norm, given any initial guess in \(D\).
    \end{enumerate}

    \textbf{Part a}
    \\
    \[
        \begin{split}
            G(\vec{x}) &= L(\vec{x})\\
            &= 
            \begin{bmatrix}
                5x_1^2 - x_2^2 + (2x_1)(x_1 - x_1) + (-2x_2)(x_2 - x_2)\\
                x_2 - (\sin(x_1) + \cos(x_2))/4 + (-\cos(x_1)/4)(x_1 - x_1) + (1 + \sin(x_2))/4(x_2 - x_2)
            \end{bmatrix}
        \end{split}
    \] 
    \textbf{Part b}
    \\

\end{homeworkProblem}

\begin{homeworkProblem}
    Use two iterations of Newton's method with initial guess \(vec{0}\) to approximate the solution to
    \[
        \begin{split}
            3x - \cos(yz) - 1/2 &= 0\\
            4x^2 - 625y^2 + 2y - 1 &= 0\\
            e^{-xy} + 20z + \frac{10\pi - 3}{3} &= 0
        \end{split}
    \]

    \textbf{Solution}
    \\

    \[
        \begin{split}
            J &=
            \begin{bmatrix}
                3 & -z\sin(yz) & -y\sin(yz)\\
                8x & 1250y + 2 & 0\\
                -ye^{-xy} & -xe^{-xy} & 20
            \end{bmatrix}
        \end{split}
    \]

    First iteration
    \[
        \begin{split}
            J(\vec{0}) &=
            \begin{bmatrix}
                3 & 0 & 0\\
                0 & 2 & 0\\
                0 & 0 & 20
            \end{bmatrix}
            \\
            J(\vec{0})\vec{y} &= -\vec{F}(\vec{0})\\
            \begin{bmatrix}
                3 & 0 & 0\\
                0 & 2 & 0\\
                0 & 0 & 20
            \end{bmatrix}
            \vec{y} &= -\vec{F}(\vec{0})
            \\
            \begin{bmatrix}
                3 & 0 & 0\\
                0 & 2 & 0\\
                0 & 0 & 20
            \end{bmatrix}
            \vec{y} &= -
            \begin{bmatrix}
                3/2\\
                1\\
                -\frac{10\pi}{3}
            \end{bmatrix}
            \\
            \vec{y} &= 
            \begin{bmatrix}
                9/2\\
                2\\
                -\frac{200\pi}{3}
            \end{bmatrix}
            \\
            \vec{x}^{(1)} &= \vec{x}^{(0)} + \vec{y}
            \\
            &= 
            \begin{bmatrix}
                9/2\\
                2\\
                -\frac{200\pi}{3}
            \end{bmatrix}
        \end{split}
    \]

    Second iteration
    \[
        \begin{split}
            J(\vec{x}^{(1)}) &=
            \begin{bmatrix}
                3 & 181.38 & -1.73205\\
                9 & 2502 & 0\\
                -.000247 & -.0005555 & 20
            \end{bmatrix}
            \\
            J(\vec{x}^{(1)})\vec{y} &= -\vec{F}(\vec{x}^{(1)})\\
            \begin{bmatrix}
                3 & 181.38 & -1.73205\\
                9 & 2502 & 0\\
                -.000247 & -.0005555 & 20
            \end{bmatrix}
            \vec{y} &= -\vec{F}(\vec{x}^{(1)})\\
            \\
            \begin{bmatrix}
                3 & 181.38 & -1.73205\\
                9 & 2502 & 0\\
                -.000247 & -.0005555 & 20
            \end{bmatrix}
            \vec{y} &= -
            \begin{bmatrix}
                3\\
                -2416\\
                198.912
            \end{bmatrix}
            \\
            \vec{y} &= 
            \begin{bmatrix}
                9/2\\
                2\\
                -\frac{200\pi}{3}
            \end{bmatrix}
            \\
            \vec{x}^{(1)} &= \vec{x}^{(0)} + \vec{y}
            \\
            &= 
            \begin{bmatrix}
                9/2\\
                2\\
                -\frac{200\pi}{3}
            \end{bmatrix}
        \end{split}
    \]


\end{homeworkProblem}

%\pagebreak
%\section{Appendix}

\end{document}
