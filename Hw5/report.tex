\documentclass{article}

\usepackage{fancyhdr}
\usepackage{extramarks}
\usepackage{amsmath}
\usepackage{amsthm}
\usepackage{amsfonts}
\usepackage{tikz}
\usepackage[plain]{algorithm}
\usepackage{algpseudocode}
\usepackage{listings}
\usepackage{enumerate}
\lstset{breaklines=true}

\usetikzlibrary{automata,positioning}

%
% Basic Document Settings
%

\topmargin=-0.45in
\evensidemargin=0in
\oddsidemargin=0in
\textwidth=6.5in
\textheight=9.0in
\headsep=0.25in

\linespread{1.1}

\pagestyle{fancy}
\lhead{\hmwkAuthorName}
\chead{\hmwkClass\ (\hmwkClassInstructor): \hmwkTitle}
\rhead{\firstxmark}
\lfoot{\lastxmark}
\cfoot{\thepage}

\renewcommand\headrulewidth{0.4pt}
\renewcommand\footrulewidth{0.4pt}

\setlength\parindent{0pt}

%
% Create Problem Sections
%

\newcommand{\enterProblemHeader}[1]{
    \nobreak\extramarks{}{Problem \arabic{#1} continued on next page\ldots}\nobreak{}
    \nobreak\extramarks{Problem \arabic{#1} (continued)}{Problem \arabic{#1} continued on next page\ldots}\nobreak{}
}

\newcommand{\exitProblemHeader}[1]{
    \nobreak\extramarks{Problem \arabic{#1} (continued)}{Problem \arabic{#1} continued on next page\ldots}\nobreak{}
    \stepcounter{#1}
    \nobreak\extramarks{Problem \arabic{#1}}{}\nobreak{}
}

\setcounter{secnumdepth}{0}
\newcounter{partCounter}
\newcounter{homeworkProblemCounter}
\setcounter{homeworkProblemCounter}{1}
\nobreak\extramarks{Problem \arabic{homeworkProblemCounter}}{}\nobreak{}

%
% Homework Problem Environment
%
% This environment takes an optional argument. When given, it will adjust the
% problem counter. This is useful for when the problems given for your
% assignment aren't sequential. See the last 3 problems of this template for an
% example.
%
\newenvironment{homeworkProblem}[1][-1]{
    \ifnum#1>0
        \setcounter{homeworkProblemCounter}{#1}
    \fi
    \section{Problem \arabic{homeworkProblemCounter}}
    \setcounter{partCounter}{1}
    \enterProblemHeader{homeworkProblemCounter}
}{
    \exitProblemHeader{homeworkProblemCounter}
}

%
% Homework Details
%   - Title
%   - Due date
%   - Class
%   - Section/Time
%   - Instructor
%   - Author
%

\newcommand{\hmwkTitle}{Homework\ \#5}
\newcommand{\hmwkDueDate}{February 18, 2015}
\newcommand{\hmwkClass}{Numerical Analysis}
\newcommand{\hmwkClassTime}{MWF 2:15}
\newcommand{\hmwkClassInstructor}{Professor Mohler}
\newcommand{\hmwkAuthorName}{Rick Sullivan}

%
% Title Page
%

\title{
    \vspace{2in}
    \textmd{\textbf{\hmwkClass:\ \hmwkTitle}}\\
    \normalsize\vspace{0.1in}\small{Due\ on\ \hmwkDueDate}\\
    \vspace{0.1in}\large{\textit{\hmwkClassInstructor\ \hmwkClassTime}}
    \vspace{3in}
}

\author{\textbf{\hmwkAuthorName}}
\date{}

\renewcommand{\part}[1]{\textbf{\large Part \Alph{partCounter}}\stepcounter{partCounter}\\}

%
% Various Helper Commands
%

% Useful for algorithms
\newcommand{\alg}[1]{\textsc{\bfseries \footnotesize #1}}

% For derivatives
\newcommand{\deriv}[1]{\frac{\mathrm{d}}{\mathrm{d}x} (#1)}

% For partial derivatives
\newcommand{\pderiv}[2]{\frac{\partial}{\partial #1} (#2)}

% Integral dx
\newcommand{\dx}{\mathrm{d}x}

% Alias for the Solution section header
\newcommand{\solution}{\textbf{\large Solution}}

% Probability commands: Expectation, Variance, Covariance, Bias
\newcommand{\E}{\mathrm{E}}
\newcommand{\Var}{\mathrm{Var}}
\newcommand{\Cov}{\mathrm{Cov}}
\newcommand{\Bias}{\mathrm{Bias}}

\begin{document}

\maketitle

\pagebreak

\begin{homeworkProblem}
    \begin{enumerate}[a)]
        \item Let \(T\) be an \(n \times n\) matrix with \(\rho(T) < 1\). 
            Show that \(\left(I-T\right)^{-1} = I + T + T^2 + T^3 + T^4 + \dots\) 
            (You may assume that both the inverse on the left and the infinite sum on the right are well defined, which happens with \(\rho(T) < 1\)).

        \item Use part a) to verify that \(\vec{x} = (1 + T + T^2 + T^3 + \dots)\vec{c}\) satisfies the fixed point equation
            \[
                \begin{split}
                    \vec{x} &= T\vec{x} + \vec{c}
                \end{split}
            \]
    \end{enumerate}

    \textbf{Part a}
    \\

    If \(\rho(T) < 1\), the largest absolute eigen value in \(T\) is less than 1. 
    \(I -T\) then reverses the signs of each element in \(T\) that does not lie on the diagonal. 
    Every \(a_{ii}\) for \(i\) in the matrix then becomes \(1 - a_{ii}\).
    \\

    Multiplying both sides by \(I-T\)
    \[
        \begin{split}
            (I-T)^{-1}(I-T) &= (I + T + T^2 + ..)(I-T)\\
            I &= I(I-T) + T(I-T) + T^2(I-T)\\
            &= I-T + T-T^2 + T^2 - T^3 + ...\\
            I &= I
        \end{split}
    \]

    Thus the statement holds.
    \\

    \textbf{Part b}
    \\

%    According to a theorem covered in class, \(\vec{x}\) approaches the unique solution of \(\vec{x} = T\vec{x} + \vec{c}\) if \(\rho(T) < 1\).
%    It's given in part a) that \(\rho(T) < 1\).
    Substituting for \(\vec{x}\)

    \[
        \begin{split}
            (1 + T + T^2 + T^3 + \dots)\vec{c} &= T(1 + T + T^2 + \dots)\vec{c} + \vec{c}\\
            1 + T + T^2 + T^3 + \dots &= T(1 + T + T^2 + \dots) + 1\\
            T + T^2 + T^3 \dots &= T(1 + T + T^2 + \dots)\\
            T + T^2 + T^3 \dots &= T + T^2 + T^3 + \dots\\
        \end{split}
    \]

    The approximation will approach the unique solution because \(\rho(T) < 1\).
    
\end{homeworkProblem}
\begin{homeworkProblem}
    Let \(A\) be a diagonally dominant \(n \times n\) matrix and \(T\) be the matrix constructed from \(A\) corresponding to Jacobi's method. Show that \(||T||_\infty < 1\).
    \\

    \textbf{Solution}
    \\

    A diagonally dominant matrix has \(|A_{ii}| \geq \sum_{j!=i} |A_{ij}|\) for all \(i\). Therefore, the diagonal entries are gauranteed to have an absolute value at least equal to the cumulative magnitudes of the other values in each row. 

    When constructing \(T\) for Jacobi's method, diagonal entries are isolated on the left hand side. In this isolation, all other entries are divided by the diagonal coefficient. For example, a row representing \(a_1x_1 + a_2x_2 + \dots + a_nx_n = c\) will become

    \[
        \begin{split}
            a_ix_i &= -(a_1x_1 + \dots a_nx_n) + c\\
            x_i &= (1/a_i)(-(a_1x_1 + \dots a_nx_n) + c)\\
        \end{split}
    \]

    Where \(i\) refers to the diagonal term, and the right hand sides do not have an \(i\) term. \(a_i\) is guaranteed to be at least the absolute sums of the other coefficients, so the right hand side will always be less than or equal to 1.

    The infinity norm of \(T\) is calculated by finding the maximum absolute sum of any row in \(T\). As explained above, the sum for any row will be less than or equal to 1, so \(||T||_\infty \leq 1\).

    Note that this is not \(||T||_\infty < 1\). That condition requires \(A\) to be \textit{strictly} diagonally dominant, which ensures that \(|A_{ii}| > \sum_{j!=i} |A_{ij}|\) for all \(i\). The logic proving that is the same.
\end{homeworkProblem}
\begin{homeworkProblem}
    \begin{enumerate}[a)]
        \item Find the first two iterations of the Jacobi method for the system \(A\vec{x} = \vec{b}\) using \(\vec{x}^{(0)} = \vec{0}\), where
            \[
                \begin{split}
                    A &= 
                    \begin{bmatrix}
                        4 & 1 & -1 & 1\\
                        1 & 4 & -1 & -1\\
                        -1 & -1 & 5 & 1\\
                        1 & -1 & 1 & 6\\
                    \end{bmatrix}
                \end{split}
            \]

            and

            \[
                \begin{split}
                    \vec{b} &=
                    \begin{bmatrix}
                        -2\\
                        -1\\
                        0\\
                        1\\
                    \end{bmatrix}
                \end{split}
            \]
        \item Let \(T\) be the Jacobi matrix associated with part a). Calculate \(||T||_\infty\) and use this to estimate the number of iterations required to achieve an accuracy of \(10^{-6}\) in the \(||\cdot||_\infty\) norm. Hint: use the error formula
        \[
            \begin{split}
                ||\vec{x}^{(k)} - \vec{x}||_\infty &\leq \frac{||T||_\infty^k}{1-||T||_\infty}||\vec{x}^{(1)} - \vec{x}^{(0)}||_\infty
            \end{split}
        \]
    \end{enumerate}

    \textbf{Part a}
    \\
    Isolate diagonal terms on the left hand side.
    \[
        \begin{split}
            4x_1 + x_2 - x_3 + x_4 &= -2\\
            4x_1 &= -x_2 + x_3 - x_4 - 2\\
            x_1 &= (1/4)(-x_2 + x_3 - x_4 - 2)
            \\[3ex]
            x_1 + 4x_2 - x_3 - x_4 &= -1\\
            4x_2 &= -x_1 + x_3 + x_4 -1\\
            x_2 &= (1/4)(-x_1 + x_3 + x_4 - 1)
            \\[3ex]
            -x_1 - x_2 + 5x_3 +x_4 &= 0\\
            5x_3 &= x_1 + x_2 - x_4\\
            x_3 &= (1/5)(x_1 + x_2 - x_4)
            \\[3ex]
            x_1 - x_2 + x_3 + 6x_4 &= 1\\
            6x_4 &= -x_1 + x_2 - x_3 + 1\\
            x_4 &= (1/6)(-x_1 + x_2 - x_3 + 1)
        \end{split}
    \]

    Applying two iterations
    \[
        \begin{split}
            x_1^{(1)} &= (1/4)(-0 + 0 - 0 - 2)\\
            &= -1/2\\
            x_2^{(1)} &= (1/4)(-(-1/2) + 0 + 0 -1)\\
            &= -1/8\\
            x_3^{(1)} &= (1/5)(-1/2 -1/8 - 0)\\
            &= -1/8\\
            x_4^{(1)} &= (1/6)(1/2 -1/8 +1/8+1)\\
            &= 1/4
            \\[3ex]
            x_1^{(2)} &= (1/4)(1/8 -1/8 - 1/4 - 2)\\
            &= (1/4)(-9/4)\\
            &= -9/16\\
            x_2^{(2)} &= (1/4)(9/16 - 1/8 + 1/4 -1)\\
            &= (1/4)(9/16 - 7/8)\\
            &= (1/4)(-5/16)\\
            &= -5/64\\
            x_3^{(2)} &= (1/5)(-9/16 - 5/64 - 1/4)\\
            &= (1/5)(-57/64)\\
            &\approx -.178125\\
            x_4^{(2)} &= (1/6)(9/16 -5/64 + 0.178215+ 1)\\
            &\approx 0.27708
        \end{split}
    \]
    
    \textbf{Part b}
    \\

    \(T\) from part a) is
    \[
        \begin{split}
            T &= 
            \begin{bmatrix}
                0 & -1/4 & 1/4 & -1/4\\
                -1/4 & 0 & 1/4 & 1/4\\
                1/5 & 1/5 & 0 & -1/5\\
                -1/6 & 1/6 & -1/6 & 0
            \end{bmatrix}
            \\
            ||T||_\infty &= \max(3/4, 3/4, 3/5, 3/6)\\
            &= 3/4
        \end{split}
    \]

    Applying the error formula with the given accuracy,
    \[
        \begin{split}
            ||\vec{x}^{(k)} - \vec{x}||_\infty \leq \frac{||T||_\infty^k}{1-||T||_\infty}||\vec{x}^{(1)} - \vec{x}^{(0)}||_\infty &\leq 10^{-6}
            \\
            \frac{||T||_\infty^k}{1-3/4}||\vec{x}^{(1)}||_\infty &\leq 10^{-6}
            \\
            \frac{(3/4)^k}{1/4}1/2 &\leq 10^{-6}
            \\
            k &\approx 50.433
        \end{split}
    \]

    So we need a minimum of 51 iterations to guarantee our estimation is within \(10^{-6}\) of the solution.

\end{homeworkProblem}


%\pagebreak
%\section{Appendix}

\end{document}
